\title{Epileptic Seizure Detection from EEG Signals with Deep Belief Networks}
\author{Tuan Nguyen}
\date{\today}

\documentclass[12pt]{article}

\usepackage{url}

\begin{document}
\maketitle

\section{Domain Background}

\noindent
The domain of interest of this project belongs to physiological data such as electroencephalography (EEG), magnetoencephalography (MEG), electrocardiography (ECG) and the recorded signals from wearable devices. This project focuses on the EEG signals, which capture activities of brain neurons during a period of time. Different kinds of EEG data has been recorded from humans, for instance from those at rest, sleep~\cite{langkvist2012sleep}, during some periods of specific cognitive activity, or from patients with diseases such as Alzheimer's, Parkinson's, depression and epileptic seisures (\cite{andrzejak2001indications} and references thereof).

This project studies the classification problem on an EEG data set recorded from healthy volunteers and patients having epileptic seizures; solutions for this problem could be used to assist with detecting patients with the disease from their brain activity signals. Previous work on this data set focused mainly on extracting hand-crafted features to be used for classification. As examples, Nigam and Graupe \cite{nigam2004neural} extracted two the spike amplitudes and frequency of the signals, feeding them into a neural network for classification; Guler et al.~\cite{guler2005recurrent} applied wavelet transform to the signal to extract features, and classification was performed using neuro-fuzzy system; \cite{kannathal2005entropies} extracted entropy-based features for classification. This project, on the other hand, aims at learning features representing the EEG signals automatically using Deep Belief Networks (DBNs)~\cite{hinton2006reducing} which will then be used with popular classification algorithms.

\section{Problem Statement}

This project aims to classifies an input EEG segment into 

======
The aim of this project is to study the effectiveness of combining different types of feature extraction mechanisms and classification algorithms on the above mentioned EEG data set~\cite{andrzejak2001indications}. In particular, it considers using raw features, hand-crafted features and features automatically learned using Deep Belief Networks~\cite{hinton2006reducing} as input to popular machine learning classifiers. The goal is to classify segments of EEG signals into one of the five target classes of subjects.

Much work has been done deriving hand-crafted features for EEG data sets (including for the one of interest in this project), for instance information about signal spikes~\cite{nigam2004neural}, geometric and shape information~\cite{wulsin2011modeling}. While many of these designed features already provided high accuracy results, they were used for the binary classification problem only (i.e., detecting whether a signal segment indicates seizure or not). Their effectiveness in distinguishing similar classes such as healthy subjects with open and closed eyes, and therefore in classifying the five target classes, is unclear. This project first aims to answer this question.

The second objective of this project is to examine the power of Deep Belief Networks in extracting features for the data set. While this network has been used to reduce data dimensionality~\cite{hinton2006reducing} and extract features for EEG data sets (c.f. ~\cite{langkvist2012sleep}, \cite{wang2013modeling}, \cite{wulsin2011modeling}), to the best of my knowledge it has not been applied to the data set provided by Andrzejak et al.~\cite{andrzejak2001indications}. It is therefore unclear whether Deep Belief Networks can be an effective tool in learning features for this data set.

Finally, the three types of features will be tested with several classifiers, in particular the Decision Trees, K-Nearest Neighbors, Support Vector Machines and Feedforward Multi-layer Neural Networks. The goal is that given a segment of EEG signals, the output will be one of the five sets of subjects to which the segment belongs. The accuracy of different combinations of feature sets and classifiers is the main performace measure.

\section{Data Sets and Inputs}

The data set used in this study is provided with the work by Andrzejak et al.~\cite{andrzejak2001indications}, recording brain electrical activity of five sets of human volunteers:
\begin{itemize}
\item Set A: healthy volunteers in relaxed state with eyes open.
\item Set B: healthy volunteers in relaxed state with eyes closed.
\item Set D: epilepsy patients during seizure-free periods; the electrodes were implanted to record brain activity from within the epileptogenic zone.
\item Set C: epilepsy patients during seizure-free periods; the electrodes were implanted from the opposite hemisphere of the brain (compared to those in Set D).
\item Set E: same patients with sets C and D but activity were recorded from all electrodes during active seizures.
\end{itemize}

For each set of volunteers above, the data set contains 100 single-channel EEG segments of 23.6-sec duration. For this project each of these segments will be further divided into smaller segments of 1-sec durations, which are considered stationary for EEG data~\cite{nigam2004neural}. Each 1-sec duration become a sample, and the corresponding volunteer set becomes the target to be classified.

\section{Solution Statements}

The project has two main parts: feature extraction/learning and classification, described briefly in the following subsections.

\subsection{Feature extraction}
The following three types of features are to be used as features of the classifiers: (1) raw features (i.e., the EEG signals), (2) hand-crafted features and (3) features exatracted using Deep Belief Networks.

\begin{itemize}

\item Raw features: the EEG signals are used directly as features.

\item Hand-crafted features: the EEG signal of each subject will first be divided into segments of 1 second; as suggested by Nigam and Graupe~\cite{nigam2004neural}, the EEG signals is considered stationary within these segments. Several features will then be extracted from each segment:
      \begin{itemize}
      \item Summary statistics of segments: min, 25\% percentile, mean, 75\% percentile, max, median.
      \item Spike based features~\cite{nigam2004neural}: maximum spike amplitude and the number of spikes in each segment.
      \item Features described in~\cite{wulsin2011modeling}: area under the curve, normalized decay of data that is increasing or decreasing, line length of the signal, mean energy of the data, peak amplitude, valley amplitude, normalized peak number, peak variation, root mean square and the wavelet energy of the data.
      \end{itemize}

\item Automatic features: The last type of features are to be learned using Deep Belief Networks (DBNs)~\cite{hinton2006reducing}. The idea of using DBNs for feature extraction has been investigated for different physicological data sets in previous work such as ~\cite{langkvist2012sleep} and ~\cite{wulsin2011modeling}, but not for this data set (to the best of my knowledge). I intend to construct Deep Belief Networks from a stack of restricted Boltzmann machines and train them on the set of 1-sec segments mentioned above, resulting in features that can then be used for classification. The training phase would be done using contrastive divergence as described by Hinton~\cite{hinton2006training}.

\end{itemize}

\subsection{Classification}

The three types of features extracted and learned are then used as input features to several classifiers. The following classifiers are considered: Decision Trees, K-Kearest Neighbors, Support Vector Machines and Feedforward Multi-layer Neural Networks (potentially with Dropout layers~\cite{srivastava2014dropout}).

\section{Benchmark Model}

There were several work that aimed to detect seizure patients from the EEG signals in this data set using different hand-chosen features (c.f.~\cite{nigam2004neural}, \cite{kabir2016epileptic}) with excellent accuracy performance. None of them, however, attempted to recognize all five classes of EEG signals from different group of volunteers. On a different data set, Wulsin et al.~\cite{wulsin2011modeling} learned features of second-long segments of EEG signals using DBNs and classified them into ``clinically significant'' EEG classes. Unfortunately, the results from these work cannot be used to compare directly with that of this project. This project, however, aims to compare different combination of feature extraction mechanisms and classification algorithms on the data set~\cite{andrzejak2001indications}, which is an interesting problem by itself.

\section{Evaluation Metrics}

The accuracy (i.e., the percentage of samples with correctly predicted target class) in predicting the volunteer classes for 1-sec segments is the main evaluation metric that will be used for this classification task.

\section{Project Design}

\subsection{Step 1: Data Preprocessing}

Use the data from UCI Seizure~\cite{UCISeizure}

Split into train-test (80/20)

Normalize the inputs in the training set, each input vector contains value from [0,1] (Sigmoid unit will be used)

\subsection{Step 2: Design and training individual restricted Boltzman machines}

Design choices to make:
\begin{itemize}
\item Type of visible and hidden units: sigmoid
\item Configuration of machines: 
\end{itemize}

\bibliographystyle{abbrv}
\bibliography{proposal}

\end{document}
